\section{Introduction}
Bitcoin is a relatively young form of digital currency that has become increasingly popular in recent years. Its decentralized and peer-to-peer nature, in addition to its financial transaction security guarantees, makes it very appealing to alternative systems. However, providing adequate user anonymity when making transactions remains an open problem that has motivated a significant body of work studying alternative constructions, protocol modifications, and general studies on Bitcoin anonymity. Of the most prominent deanonymization techniques are network- and protocol-level analysis. Kaminsky \cite{kaminsky} pioneered network-level attacks on client anonymity by eavesdropping on network activity and associating the flow of transactions with the IP addresses from which they originated. This type of attack served to circumvent the now standard practice of generating fresh key pairs and shadow change (output collection) addresses for every transaction so as to ensure unlinkability among users and their transactions. 

Protocol-level studies and attacks, such as those performed by TODO, rely on the flow of transactions and other side-channel information in order to deanonymize clients. Even with the usage of anonymity layers such as Tor to hide their original IP addresses, such graph-based techniques are highly effective at linking transactions to their original users. Using eWallets or mixing services to further break the link between users and their transactions is often recommended, but there are numerous problems associated with these partial solutions. Perhaps most importantly, such services need to be trusted to not disclose the identity of their clients or simply steal a users funds. Designing mixing trustworthy and accountable mixing services has been studied several times in the literature, and the solutions to date either require modifications to the Bitcoin protocol \cite{BitterToBetter} or have not yet been deployed to be adequately assessed in practice \cite{mixcoin}. 

TODO: coinjoin/sharecoin

To circumvent the use of mixing services, others have proposed enhancements to the Bitcoin protocl that provide cryptographic guarantees of anonymity. Of these proposals, Zerocoin \cite{zerocoin} appears to be the most feasible solution to implement in practice and see widespread adoption since it builds upon Bitcoin as a backing currency. Although there have been claims that its performance, which is based on RSA accumulators and zero-knowledge proof systems with expensive generation, verification, and large proofs, will ultimately impeded its acceptance, enhancements such as PinocchioCoin \cite{pinocchio} have been proposed to allieviate such issues. PinocchioCoin provides the same functionality as Zerocoin but uses more efficient pairing-based primitives and the Pinocchio verificable computation scheme to replace expensive zero-knowledge proofs. Also, while briefly discussed in their respective publications, each of these ``anonymous'' coin schemes that build upon Bitcoin require a network-layer anonymizing layer such as Tor to prevent network-level attacks.

Based on these observations, it is clear that client anonymity guarantees at the network layer is fundamental for client anonymity at the level of the Bitcoin protocol. However, it would ideal if clients did not have to install or rely upon separate networking software to achieve this anonymity. To this end, we propose Bitcoin Boomerang, a peer-to-peer mixnet \emph{built into} the Bitcoin protocol to provide virtually the same anonymity guarantees as network-layer services such as Tor without the same vulnerabilities (e.g., timing side channel attacks). 

In this document we describe the detail the design of Bitcoin Boomerang, which shall henceforth be referred to as Boomerang for simplicity, and discuss the anonymity guarantees, expected performance, and preliminary implementation and simulation results that support our claims. The remainder of this document is outlined as follows. Section 2 provides an overview of the scheme, section 3 provides a detailed discussion of the Boomerand design, section 4 presents an analysis of the security and anonymity that Boomerang enables, and finally, section 5 discusses the expected performance of Bitcoin when using Boomerang for anonymity. 



