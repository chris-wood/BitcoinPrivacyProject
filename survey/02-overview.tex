\section{Bitcoin Preliminaries} \label{sec:preliminaries}

To appreciate the simplicity of the anonymity flaws in Bitcoin system, all that is required is an understanding of transaction generation and verification. To that end, we now provide an overview of the Bitcoin transaction system and these two procedures, which are presented with an emphasis on the protocol-level and structural properties that render the currency vulnerability to many deanonymization attacks.

\subsection{Transaction Generation}

In what follows we distill a description of the Bitcoin system and underlying protocol from \cite{bitcoin}; interested readers may acquire more specific details therein if required. To reiterate, Bitcoin is a distributed, decentralized form of cryptocurrency. Accordingly, this enables all (digitally signed) transactions between two parties to be conducted in a peer-to-peer fashion without the inclusion of a trusted third party, such as a bank or other financial institution. This form of decentralized exchange comes at a price, however, as there must be some way to prevent users from \emph{double spending}, or using the same funds to simultaneously pay multiple parties. Bitcoin achieves this property by relying on its users to construct a history for every transaction that takes place in the system, which is referred to as the system ledger. If a majority of the users accept the validity of a particular transaction based on provided cryptographic hash digest (to be discussed later), or a set of transactions, it is ``confirmed'' and the global ledger of the system is affirmatively updated. This validation hash digest, referred to as a hash-based proof-of-work, is the fundamental technique underlying the correctness and accuracy of the ledger contents. By the properties of the hash function, the system ledger cannot be changed without breaking the function (i.e., finding a collision) or re-doing the proof-of-work \emph{faster} than honest nodes working to verify and update the ledger, which is computationally infeasible for small groups of nodes. Therefore, so long as a majority of the Bitcoin users are honest, the system history is accepcted as correct and all signed transactions are considered valid, which prevents double spending by potentially malicious users (the ledger can be examined to see if funds have been previously spent elsewhere). 

While the above scheme is semantically correct and provides strong guarantees that all financial transactions are valid, there are inherent limitations in the amount of user privacy and anonymity that can be achieved in Bitcoin. In order to adequately define these limitations, we first describe how Bitcoin transactions are generated and how the system ledger is maintained. For simplicity, consider the scenario in which user $A$ wants to send $N$ Bitcoins (BTCs) to user $B$. Rather than identify users by name, Bitcoin uses \emph{pseudonomous addresses} that are tied to specific users to use in such transactions. Denote $\mathsf{addr}_A$ and $\mathsf{addr}_B$ as the addresses of users $A$ and user $B$ used in this transaction, respectively. It is often convenient to think of Bitcoin addresses as public keys $\mathsf{pk}_A$ and $\mathsf{pk}_B$, and as such there are corresponding private keys, which we denote as $\mathsf{sk}_A$, and $\mathsf{sk}_B$, respectively.

Structurally, a transaction $T$ is a tuple comprised of the \emph{source} transactions which supplied the funds necessary to make this transaction, denoted as $\mathsf{source}$, the (public) address of the recipient, $\mathsf{addr}_B$, the amount of BTCs to send, $N$, and a digital signature of these three properties, $\mathsf{Sign}_{\mathsf{sk_A}}({\mathsf{source}, \mathsf{addr}_B, N})$. In other words, we have 
\begin{align*}
T = (\mathsf{source}, \mathsf{addr}_B, N, \sigma), 
\end{align*}
where $\sigma = \mathsf{Sign}_{\mathsf{sk_A}}({\mathsf{source}, \mathsf{addr}_B, N})$. Note that this signature is embedded in $T$ so that any other Bitcoin user may verify the validity of the content using $\mathsf{pk}_A$, which is implicitly tied to the transaction. Also note that $\mathsf{source}$ need not be a single transaction; user $A$ is free to use multiple transactions in order to fund their transaction to $B$. In addition to the $N$ BTC transfer from $A$ to $B$, there is often $C$ BTC amount specified in the transaction for a particular change address, where $C$ denotes the amount of change, in BTCs, that will be given to this address as a result of the transaction. It is not required that the address to which $C$ is addressed is the same as the address of $A$, though this sometimes happens in practice. One final piece of the output of a transaction is a transaction fee, which is a small subset of $N$ that is granted to the mining node that successfully verifies this transaction. We discuss this verification process in the following section.

Figure \ref{fig:transaction-io} illustrates the input and output relation of our transaction from $A$ to $B$, and Figure \ref{fig:transaction-create} illustrates the steps used in constructing this transaction. Note that, in both cases, $\mathsf{source}$ is comprised of two transactions $T_1$ and $T_2$, where $N = \mathsf{Val}(T_1) + \mathsf{Val}(T_2)$, and the resulting transaction is denoted as $T_3$.

\begin{center}
\begin{figure}
\includegraphics[scale=0.5]{./images/transaction_io.pdf}
\caption{Visual depiction of the input and output elements of a transaction from user $A$ to user $B$.}
\label{fig:transaction-io}
\end{figure}
\end{center}

\begin{center}
\begin{figure}
\includegraphics[scale=0.4]{./images/transaction_create.pdf}
\caption{Visual depction of the steps to create a transaction $T_3$ from user $A$ to user $B$ using two input transactions, $T_1$ and $T_2$.}
\label{fig:transaction-create}
\end{figure}
\end{center}

\subsection{Transaction Verification}

After a transaction has been created, it is broadcasted in the network. In order to prevent double spending, nodes must confirm this transaction and append it to the longest chain of accepted (confirmed) transactions in the system's ledger. This procedure is based on the aforementioned proof-of-work, which works as follows. Bitcoin miners (i.e., verifying nodes) will collect unconfirmed transactions into a block or buffer, along with the longest chain of system-wide accepted transactions, and compute a Merkle hash of the transactions and digest of the chain. The output digest of this Merkle hash, referred to as the challenge $c$ in the proof-of-work protocol, is then used to find the proof $p$. Together, $c$ and $p$ have the property that, when concatenated and hashed using a cryptographically strong collision-resistant hash function $H$, the leading $B$ bits of the output $x = H(c || p)$ are all $0$. That is, $x = 0^B\{0,1\}^{256-B}$. Given the collision resistant properties of $H$, finding a valid proof $p$ for the challenge $c$ is comptuationally difficult, thus making it highly improbable for malicious nodes to alter the contents of the ledger. Figure \ref{fig:pow} illustrates the construction of $c$ and $p$ using a previously confirmed block chain $B$.

\begin{center}
\begin{figure}
\includegraphics[scale=0.5]{./images/transaction_pow.pdf}
\caption{Proof-of-work computational procedure using the transactions of a block, the digest of the previous block, and the sampled proof $p$.}
\label{fig:pow}
\end{figure}
\end{center}

Once a miner finds a proof, it is broadcasted to the other nodes in the network along with the input transactions used by the miner. Other nodes can then easily recompute the challenge $c$ and verify the correctness of $p$. Once verified, this new transaction ``block'' is appended to the block chain which the miner used in finding the proof. Figure \ref{fig:block} illustrates a snippet of the block chain, where the challenge $c$ is the digest of the previous block and the proof $p$ are embedded in each block. Miners will continually use the longest block chain to gather and verify transaction blocks. Since there is a particular subset of BTCs in each transaction that are paid to the miner who provides the proof-of-work for a block containing that transaction, referred to as the transaction fee, miners are financially incentivized to collect more transactions into a block and continually ``mine'' for valid proofs-of-work. 

\begin{center}
\begin{figure}
\includegraphics[scale=0.5]{./images/transaction_block_pow.pdf}
\caption{A snippet of a Bitcoin transaction block chain, illustrating the groupings of transactions into a blocks, the declaration of the proof of the work $p$, and the digest of the previous block linking the blocks together.}
\label{fig:block}
\end{figure}
\end{center}

\subsection{Standard Bitcoin Anonymity Practices}
As the topic of the survey indicates, Bitcoin has serious anonymity flaws. However, there are several standard practices that Bitcoin clients and users are recommended to follow in order to improve their overall anonymity and decrease the likelihood of becoming the target of deanonymization attacks. First, clients (and users) should specify \emph{shadow addresses} to collect change from a transaction \cite{bitcoin-shadow-addresses}. Such addresses are intended to be distinct from the user's address used at the time of the transaction. Furthermore, since change need not always be returned to the user who provided the BTC funds, this disjoint address helps obfuscate the link between the address and the original user, thus helping improve overall anonymity. Secondly, it is recommended that all users continually generate new transaction addresses and corresponding public and private key pairs in order to deter attacks that stem from address re-use. We discuss attacks of this nature in the following sections. Thirdly, it is often recommended that users connect to the Bitcoin network using an anoymizing layer such as Tor \cite{tor,bitcoin-tor-wiki} so as to obfuscate their network-layer identities. Beyond these simple practices, there does not exist any other standardized techniques that may be used to help improve user anonymity.

