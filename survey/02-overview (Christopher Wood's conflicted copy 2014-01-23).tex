\section{Bitcoin Overview and Privacy Limitations}
% TODO: specific discussion of aspects of bitcoin that pertain to privacy/anonymity

Bitcoin is a distributed, decentralized form of cryptocurrency. Accordingly, this enables all (digitally signed)  transactions between two parties to be conducted in a peer-to-peer fashion without the inclusion of a trusted third party, such as a bank or other financial institution. This form of decentralized exchange comes at a price, however, as there must be some way to prevent users from \emph{double spending}, or using the same funds to simultaneously pay multiple parties. Bitcoin achieves this property by relying on its users to construct a history for every transaction that takes place in the system. If a majority of the users accept the validity of a particular transaction, or a set of transactions, the global history of the system is affirmatively updated and ``confirmed'' via a cryptographic hash digest that all users agree upon. This hash digest, referred to as a hash-based proof-of-work, is what constitutes the validity of the system. By the properties of the underlying hash function, the history of the system cannot be changed without breaking the function (i.e., finding collisions) or re-doing the proof-of-work, which is computationally infeasible for small groups of nodes. Therefore, so long as a majority of the Bitcoin users are honest, the system history is deemed correct and thus all signed transactions are verified, preventing double spending by potentially malicious users participating in direct, peer-to-peer transactions. 

Unfortunately, while the above scheme is semantically correct and provides strong guarantees that all financial transactions are valid, there are inherent limitations in the amount of user privacy and anonymity that can be achieved in Bitcoin. In order to adequately define these limitations, we first describe how Bitcoin transactions are generated and how the system history is maintained. For simplicity, consider the scenario in which user $A$ wants to send $N$ BTCs (Bitcoins) to user $B$. Rather than identify users by name, Bitcoin uses \emph{addresses} that are tied to specific users to use in such transactions. Denote $\mathsf{addr}_A$ and $\mathsf{addr}_B$ as the addresses of users $A$ and user $B$ used in this transaction. It is often convenient to think of Bitcoin addresses as public keys $\mathsf{pk}_A$ and $\mathsf{sf}_B$, and as such there are corresponding private keys, which we denote as $\mathsf{sk}_A$, and $\mathsf{sk}_B$, respectively.

Structurally, a transaction $T$ is a tuple comprised of the \emph{source} transactions which supplied the funds necessary to make this transaction, denoted as $\mathsf{source}$, the (public) address of the recipient, $\mathsf{addr}_B$, the amount of BTCs to send, $N$, and a digital signature of these three properties, $\mathsf{Sign}_{\mathsf{sk_A}}({\mathsf{source}, \mathsf{pk}_B, N})$. In other words, we have 
\begin{align*}
T = (\mathsf{source}, \mathsf{pk}_B, N, \sigma), 
\end{align*}
where $\sigma = \mathsf{Sign}_{\mathsf{sk_A}}({\mathsf{source}, \mathsf{pk}_B, N})$. Note that this signature is embedded in $T$ so that any other Bitcoin user may verify the validity of the content using $\mathsf{pk}_A$. Also note that $\mathsf{source}$ need not be a single transaction; user $A$ is free to use multiple transactions in order to fund their transaction to $B$. In addition to the $N$ BTC transfer from $A$ to $B$, there is often $C$ BTC amount specified in the transaction for a particular address, where $C$ denotes the amount of change that will be given to this address as a result of the transaction. It is not required that the address to which $C$ is addressed is the same as the address of $A$, though this often happens in practice. Figure \ref{fig:transaction-io} illustrates the input and output relation of our transaction from $A$ to $B$, and Figure \ref{fig:transaction-create} illustrates the steps used in constructing this transaction. Note that, in both cases, $\mathsf{source}$ is comprised of two transactions $T_1$ and $T_2$, and the resulting transaction is denoted as $T_3$.

\begin{center}
\begin{figure}
\includegraphics[scale=0.5]{./images/transaction_io.pdf}
\caption{Visual depiction of the input and output elements of a transaction from user $A$ to user $B$.}
\label{fig:transaction-io}
\end{figure}
\end{center}

\begin{center}
\begin{figure}
\includegraphics[scale=0.4]{./images/transaction_create.pdf}
\caption{Visual depction of the steps to create a transaction $T_3$ from user $A$ to user $B$ using two input transactions, $T_1$ and $T_2$.}
\end{figure}
\end{center}


