\section{Privacy and Anonymity}
%TODO: definitions, summary of security model and attackers, and goals of the attackers
%Also, Vu needs to figure out how to do citations

%Vu - brainstorming: delete later
\subsection{Delete Me}
Users create addresses. Addresses are used in transactions.  Privacy is the deassociation of users and transactions.  Because linking addresses to transactions is trivial with block chain, privacy deals with identifying users with addresses. Specifically, given a user, the adversary must identify the transactions the user is involved with and by extension the addresses the user controls (address unlinkability).
In contrast, anonymity requires the adversary to identify users and by extension addresses from transactions. Once again, identifying addresses from transactions is trivial and therefore given an address, the adversary must identify the user.
In both cases, attacks attempt to group addresses together. In the case of privacy, by grouping addresses with a self-identified address, other transactions by the user is revealed.  In the case of anonymity, addresses are linked together to create a more detailed picture that can be subject to behavioral analysis or matched to external data.  The inability to group addresses together is address indistinguishability.


\subsection{Definitions}
\emph{Privacy} refers to the inability of an adversary $A$ to link a user $U$ with any transactions involving $U$. The Bitcoin block chain links all transactions to addresses, therefore privacy is only preserved if $A$ is unable to link $U$ to any of the addresses involved in any transaction involving $U$. The inability of $A$ to link $U$ with any address defines the property \emph{address unlinkability}. As shown below, analysis of privacy will be centered around address unlinkability.

In contrast to privacy is \emph{anonymity}, which refers to the fact that adversary $A$ should not be able to link any set of transactions to any user. Given a record of all transactions, such as the Bitcoin block chain, $A$ should not be able to identify any user. This is a much stronger security notion as it protects all users from being identified from any set of transactions. Since transactions are linked to addresses, $A$ should not be able to identify any user from a given set of addresses.  Corollary to that, given two addresses, $A$ should not be able to determine if they have a common owner.  This property is defined as \emph{address indistinguishability}.

%Vu - I may switch attack on privacy with anonymity and have anonymity first

\subsection{Attacks on Privacy}
%Current loss of privacy through voluntary identification, flow analysis, IP tracking. Needs more info
As a currency system, Bitcoin cannot have perfect privacy. Some address ownership information must be public knowledge. For instances, a store needs to have a publicly identifiable address in order to accept payment for goods or services.

As previously mentioned, Bitcoin uses a peer-to-peer network to transmit transactions. Transactions and addresses can be linked to the IP address that first transmits the transaction. A TCP/IP layer attack would include malicious nodes looking for clients listening to port TCP/8333 [6]. While proxy services like Tor can hide outbound connections, an inbound connection will be able to log a users IP address.

\subsection{Attacks on Anonymity}
%attacks on address indistinguishabiilty include multi-address transactions and shadow/change accounts
Researchers attempt to break the anonymity of Bitcoins in an attempt to study stolen Bitcoins [7][8] 
The two major heuristics for breaking address indistinguishabiilty are multi-address transactions and shadow or change accounts.
Multi-address transactions are transactions with more than one source. Currently, Bitcoin allows for users to use more than one source address in a transaction, but does not allow multiple users to pay for one transaction. For example, suppose $\mathsf{addr}_A$ has 3 Bitcoins (BTC) and $\mathsf{addr}_B$ has 2 BTC. The user uses both addresses to pay 4 BTC to $\mathsf{addr}_C$ and puts the remainder of 1 BTC to $\mathsf{addr}_D$. Only one user can be the input to any transaction, therefore in this example, $\mathsf{addr}_A$ and $\mathsf{addr}_B$ belong to the same user.

Shadow or change accounts are accounts created for change from a transaction. In the transaction above, $\mathsf{addr}_D$ is the shadow account that belongs to the same user that controls $\mathsf{addr}_A$ and $\mathsf{addr}_B$. Although not directly related to the Bitcoin system, the way Bitcoin clients handle shadow accounts can break address indistinguishability [4]. However, because shadow accounts rely on user behavior instead of an inherent property of the Bitcoin system, the shadow account heuristic is not as robust [6]. New clients obfuscate shadow accounts by issuing transactions to multiple users as well as creating multiple shadow addresses that look similar to legitimate transactions.
%[6] 6.	Dan Kaminsky. 2011. Black Ops of TCP/IP 2011. Black Hat USA 2011. http://www.slideshare.net/dakami/black-ops-of-tcpip-2011-black-hat-usa-2011
%[7] is 3.	Sarah Meiklejohn, Marjori Pomarole, Grant Jordan, Kirill Levchenko, Damon McCoy, Geoffrey M. Voelker, and Stefan Savage. 2013. A fistful of bitcoins: characterizing payments among men with no names. In Proceedings of the 2013 conference on Internet measurement conference(IMC '13). ACM, New York, NY, USA, 127-140
%[8] is 1.	Fergal Reid and Martin Harrigan. 2013. An Analysis of Anonymity in the Bitcoin System. Security and Privacy in Social Networks. Springer, New York, NY, USA, 197-223
