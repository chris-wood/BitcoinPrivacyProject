\section{Definitions and Adversarial Assumptions}
\emph{Privacy} refers to the inability of an adversary $A$ to link a user $U$ with any transactions involving $U$. The Bitcoin block chain\footnote{We use the term block chain, transaction history, and ledger interchangeably in this survey.} links all transactions to addresses, therefore privacy is only preserved if $A$ is unable to link $U$ to any of the addresses involved in any transaction involving $U$. In contrast to privacy is \emph{anonymity}, which is captured and quantified with respect to \emph{unlinkability} and \emph{address} or \emph{user profile indistinguishability} \cite{Androulaki12-privacy}. Activity unlinkability refers to the property that an adversary $A$ should not be able to link any set of transactions to any user. Given a record of transactions, perhaps acquired from the global ledger, $A$ should not be able to identify any specific user. Note that this is a much stronger security notion as it protects all users from being identified from any set of transactions. Furthermore, since transactions are linked to addresses, $A$ should not be able to identify any user from a given set of addresses; this property is referred to as address unlinkability. Similar to address unlinkability is user profile indistinguishability. In fact, one may view it as an addition to the prior definition in that user profile indistinguishability holds if, given two addresses, an adversary $A$ should not be able to determine if they have a common owner. Put another way, Bitcoin users enjoy a measure of profile indistinguishability if an adversary is not able to group the addresses or transactions based on the underlying Bitcoin users. 

An attack on privacy or anonymity is referred to as a deanonymization attack. Adversaries seeking to perform such attacks clearly have several distinct advantages. First, transactions are publicly broadcast throughout the network, and as users of the Bitcoin system or passive bystanders, they will therefore have access to this log. Additionally, adversaries may have access to the addresses associated with particular vendors that partake in transactions. That is, they may be able to identify and group transactions made by vendors or other specific users whose addresses are acquired via external means. However, for practical reasons, we also enforce that all adversaries are computationally bounded, i.e., any algorithm they may run or attack they may leverage must be carried out in polynomial time. Without this restriction it would be possible for an attacker to forge signatures, double-spend confirmed transactions by re-doing proofs-of-work, etc., among other scenarios. 

% \section{Privacy and Anonymity}
%TODO: definitions, summary of security model and attackers, and goals of the attackers
%Also, Vu needs to figure out how to do citations

%Vu - brainstorming: delete later
% \subsection{Delete Me}
% Users create addresses. Addresses are used in transactions.  Privacy is the deassociation of users and transactions.  Because linking addresses to transactions is trivial with block chain, privacy deals with identifying users with addresses. Specifically, given a user, the adversary must identify the transactions the user is involved with and by extension the addresses the user controls (address unlinkability).
% In contrast, anonymity requires the adversary to identify users and by extension addresses from transactions. Once again, identifying addresses from transactions is trivial and therefore given an address, the adversary must identify the user.
% In both cases, attacks attempt to group addresses together. In the case of privacy, by grouping addresses with a self-identified address, other transactions by the user is revealed.  In the case of anonymity, addresses are linked together to create a more detailed picture that can be subject to behavioral analysis or matched to external data.  The inability to group addresses together is address indistinguishability.

