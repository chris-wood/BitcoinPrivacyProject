\section{Introduction}
%TODO: general overview of cryptocurrency, bitcoin (why it's unique), and a summary of problems it suffers from

Electronic commerce would benefit greatly from the existence of a completely secure, private, and anonymous form of digital currency that does not rely on trusted third parties or external financial institutions to manage transactions. Motivated by this ideal type of currency, there have been many research efforts focused on generating suitable cryptography-based digital payment systems, or cryptocurrencies, such as DigiCash \cite{digicash}, E-Cash \cite{ecash}, HashCash \cite{hashcash}, Namecoin \cite{namecoin}, Peercoin \cite{peercoin}, Litecoin \cite{litecoin}, Ripple \cite{ripple}, and perhaps the most popular variant, Bitcoin \cite{bitcoin}. Each of these schemes offer different tradeoffs of security, privacy, and anonymity, and as such have varying popularity among users. However, it is the distribtued, decentralized nature of Bitcoin that has led to its widespread popularity among the general public and research communities. 

Bitcoin is distinguished from other cryptocurrencies by the fact that it does not rely on trusted third parties. Specifically, the global and publicly accessible ledger that stores records of all financial transactions, thereby serving as a verifiable history of all Bitcoin funds in circulation, is maintained by a widely distributed, peer-to-peer network of (untrusted) users. Transactions are linked to specific identities, or pseudonyms, via digital signatures used to ensure the validity of each transaction. In this context, it is often convenient to associate specific pseudonomous addresses with a single public and private key pair owned by a particular user. Unfortunately, these pseudonoyms are very weak masks for the underlying user's identity - user privacy and anonymity are still at risk even with the use of such pseudonomous identities. This is true even if a user has multiple pseudononyms and uses them with caution to deter attackers looking for such links. Consequently, user deanonymization is a major problem for Bitcoin users, and there have been several academic efforts to further the cause for Bitcoin user privacy and anonymity, including studies by Reid and Harrigan \cite{ReidHarrigan13} and Androulaki et al. \cite{Androulaki12-privacy}, and we can expect to see similar work publishing in the coming years. Elias \cite{elias} also discussed some legal, and moral, aspects of the anonymity, or lack thereof, in Bitcoin. We do not focus on such legality issues here, and merely operate under the assumption that spender anonymity is an ideal property that any currency system should have.

Currently, techniques to address such anonymity issues with Bitcoin are rather limited and include the use of Chaumian's entirely independent e-cash system \cite{chaumain}, which relies on trusted third parties, and Zerocoin \cite{zerocoin}, which achieves privacy and anonymity properties based on strong cryptographic assumptions at the protocol-level by working \emph{on top of} Bitcoin, among others. The former is not ideal for several reasons; the most significant of which is that it directly conflicts with the decentralized nature of Bitcoin. The latter technique is very young, having only been published in the past year, and is just now starting to gain considerable attention \cite{pinocchio}. 

In this work we survey Bitcoin and related forms of cryptocurrency with respect to their privacy and anonymity properties. We analyze proposed solutions and offer critical insight into the open problems and difficulties in achieving perfect privacy and anonymity with minimal resource consumption (e.g., bandwidth, CPU cycles, etc.). We hope that this survey will motivate continued research on this critical problem that has the potential to change financial instutitions and forms of currency for future generations.

The rest of this survey is outlined as follows. Section \ref{sec:preliminaries} presents the fundamentals of Bitcoin needed to understand its inherent anonymity limits. Section \ref{sec:adversaries} discusses the main properties of adversaries in deanonymization attacks, and we give an overview of such attacks and studies on Bitcoin anonymity in Section \ref{sec:attacks}. We then follow-up with proposed solutions to improve Bitcoin anonymity in Section \ref{sec:solutions}. We then conclude by elaborating on the fundamental problems uncovered throughout the development of this survey and offer a new approach to consider for Bitcoin anonymity in \ref{sec:open}, and then briefly discuss related cryptocurrencies and their anonymity properties in Section \ref{sec:related}. 


