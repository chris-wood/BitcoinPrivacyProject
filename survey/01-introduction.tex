\section{Introduction}
%TODO: general overview of cryptocurrency, bitcoin (why it's unique), and a summary of problems it suffers from

Electronic commerce would benefit greatly from the existence of a complete secure, private, and anonymous form of digital currency that did not rely on trusted third parties or external financial institutions to manage transactions. Indeed, there has been significant research invested in this problem in recent years resulting in various forms of cryptography-based digital payment systems, or cryptocurrencies for short, such as DigiCash \cite{digicash}, ecash \cite{ecash}, HashCash \cite{hashcash}, Namecoin \cite{namecoin}, Peercoin \cite{peercoin}, Litecoin \cite{litecoin}, Ripple \cite{ripple}, and perhaps the most popular variant, Bitcoin \cite{bitcoin}. Each of these schemes offer different tradeoffs of security, privacy, and anonymity, and as such have varying popularity among users. However, it is the distribtued, decentralized nature of Bitcoin that has led to its leading popularity to the general public and research communities. 

More specifically, Bitcoin is distinguished from other solutions by the fact that it does not rely on trusted third parties. Specifically, the global and publicly accessible ledger which stores records of financial transactions is maintained by a widely distributed, peer-to-peer network of (untrusted) users. Even though each transaction is linked to the public key of a particular user via a digital signature, rather than their real identity, user privacy and anonymity are still at risk because public keys must be owned by specific users. This is true even if a user has multiple public keys and uses them with caution to deter attackers looking for such links. Consequently, the privacy of Bitcoin is an open problem and illustrates the difficulty in achieving a distributed form of cryptocurrency, i.e., one that does not rely on trusted third parties, and one that provides sufficiently useful characteristics such as privacy and anonymity.

Currently, techniques to address such privacy issues with Bitcoin are rather limited and include the use of Chaumian's entirely independent e-cash system \cite{chaumain}, which relies on trusted third parties, or Zerocoin \cite{zerocoin}, which achieves privacy and anonymity at the protocol-level by working in conjunction with Bitcoin, among others. The former is not ideal for several reasons, the most significant of which is that it directly conflicts with the decentralized nature of Bitcoin. That is, reliance on trusted third parties is generally unfavorable if at all feasible. The latter technique is very young, having only been published in the past year, and is just now starting to gain considerable attention. Of course, there exists other academic efforts to further the cause for Bitcoin user privacy, including studies by Ron and Shamir \cite{Shamir13-bitcoingraph} and Androulaki et al. \cite{Androulaki12-privacy}, and we can expect to see similar work publishing in the coming years.

In this work we survey Bitcoin and related forms of cryptocurrency with respect to the security, privacy, and anonymity guarantees provided by each. We assess proposed solutions that have and have not been implemented in practice, and offer critical insight into the open problems and difficulties in achieving complete security, privacy, and anonymity with minimal resource consumption (e.g., bandwidth, computational cycles, etc.). We hope that this survey will motivate continued research on this critical problem that has the potential to change financial instutitions and forms of currency for future generations.

TODO: outline the sections here