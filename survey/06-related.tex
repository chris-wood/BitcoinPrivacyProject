\section{Related Cryptocurrencies}
Although Bitcoin is inarguably the most popular cryptocurrency to date, it is by no means the first such technology proposed. Other popular variants include DigiCash \cite{digicash}, E-Cash \cite{ecash}, HashCash \cite{hashcash}, Namecoin \cite{namecoin}, Peercoin \cite{peercoin}, Litecoin \cite{litecoin}, and Ripple \cite{ripple}. In this section we briefly describe some of these variants with respect to their anonymity properties. Our discussion begins with E-Cash \cite{ecash}.

Chaum, Fiat, and Naor offer two methods to create untraceable electronic cash. Both methods require a central bank organization.  The first method uses “untraceable coins” in that each transaction is of set amounts. Alice creates random commitments and sends them to the bank.  The bank uses a subset of those commitments to create cryptographic coins by signing them.  The bank then sends those coins to Alice after decrementing the amount from Alice’s account.  Everyone can verify the coin’s structure and the bank’s signature. Alice can then spend those coins to pay Bob.  To do so, Alice sends Bob the information on the coin as well as a zero-knowledge proof that Alice initiated the creation of the coin.  When Bob tries to redeem the coin with the bank, Bob sends both pieces of information to the bank.  The bank will not have enough information to identify the coin came from Alice unless Alice tries to double spend.  In that case, there is a high probability one of the challenges in the zero-knowledge proofs from the merchants will be different allowing the bank to reconstruct Alice’s identity.

The second method outlined is ``untraceable checks''.  The idea builds upon the “untraceable coins” idea by giving Alice has a roll of coins.  Alice can spend them by indexing the roll of coins by the purchase amount and revealing that to the merchant.  Alice is then refunded the rest of the amount by the bank by creating a separate transaction to pay herself.

Compact E-Cash builds upon the framework that Chaum creates with Untraceable Electronic Cash by creating a coin that can be used repeatedly a limited number of times.  The system is comprised of a “withdrawal protocol”, a “spending protocol”, and a “double spending” check.  The withdrawal protocol involves the user creating a private key as well as the coin’s serial number and blinding value.  The bank then signs the values and decrements the user’s account accordingly.  The spending protocol has the user give the merchant the coin’s serial number, the merchant’s random number challenge, and a double-spending value based on the user’s private key, the random number challenge, and the blinding value.  The user also gives the merchant two non-interactive proofs.  The first proof is that the committed coin was signed by the bank.  The second proof verifies that the coin’s serial number and double spending number correspond to the commitment as well.  The merchant then reveals this information to the bank for payment.  In order to protect the user’s privacy, the serial number that is revealed to the merchant is encrypted using the user’s private key.  In order to prevent the user from cheating, if the same coin is used over the limited number of uses, the bank is able to infer the secret key of the user, decrypt the serial number of the coin, and identify the user.  This also means that the user can re-use coins a number of times without the bank being able to identify the user.
 
The Compact E-Cash system allows for users to create coins that can be used multiple times.  The coins do not reveal the spending habits of the user if the user does not try to cheat an re-use the same coin too many times.  The drawbacks are that there must be a central bank to issue coins and verify transactions.  Also, each time a coin is used is a separate transaction even if the user spends multiple coins with the same merchant.

% http://eprint.iacr.org/2012/596.pdf